\documentclass[a4,10pt]{report}
\usepackage[latin1]{inputenc}
\usepackage{amsmath}
\usepackage{amsmath,bm}
\usepackage{amsthm}
\usepackage{mathtools}
\usepackage{gensymb}
\usepackage{amsfonts}
\usepackage{amssymb}
\usepackage{graphicx}
\usepackage{array}
\usepackage{booktabs}
\usepackage{hyperref}
\usepackage{multicol}
\usepackage[margin=0.5in]{geometry}
\usepackage{karnaugh-map}
\usepackage[framemethod=tikz]{mdframed}
\usepackage{makecell}
\newcommand{\myvec}[1]{\ensuremath{\begin{pmatrix}#1\end{pmatrix}}}
\let\vec\mathbf
\newcommand{\mydet}[1]{\ensuremath{\begin{vmatrix}#1\end{vmatrix}}}
\providecommand{\mbf}{\mathbf}
\providecommand{\pr}[1]{\ensuremath{\Pr\left(#1\right)}}
\providecommand{\qfunc}[1]{\ensuremath{Q\left(#1\right)}}
\providecommand{\sbrak}[1]{\ensuremath{{}\left[#1\right]}}
\providecommand{\lsbrak}[1]{\ensuremath{{}\left[#1\right.}}
\providecommand{\rsbrak}[1]{\ensuremath{{}\left.#1\right]}}
\providecommand{\brak}[1]{\ensuremath{\left(#1\right)}}
\providecommand{\lbrak}[1]{\ensuremath{\left(#1\right.}}
\providecommand{\rbrak}[1]{\ensuremath{\left.#1\right)}}
\providecommand{\cbrak}[1]{\ensuremath{\left\{#1\right\}}}
\providecommand{\lcbrak}[1]{\ensuremath{\left\{#1\right.}}
\providecommand{\rcbrak}[1]{\ensuremath{\left.#1\right\}}}
\begin{document}
\raggedright{\includegraphics[scale=0.07]{logo.jpg}}\hspace{12.425cm}\raggedleft FWC22025\vspace{2mm}\\
\centering\Large\textbf{OPTIMIZATION-ADVANCED}\vspace{5mm}\\
\begin{multicols}{2}
\centering \large\textsc{C}\footnotesize\textsc{ONTENTS}\vspace{5mm}\\
\raggedright\large\textbf{1\hspace{1cm}Problem}\hspace{5.2cm}1\vspace{5mm}\\
\raggedright\large\textbf{2\hspace{1cm}Solution}\hspace{5.25cm}1\vspace{5mm}\\
\centering \large\textsc{1  P}\footnotesize\textsc{ROBLEM}\vspace{5mm}\\
Show that the rectangle of maximum area that can be inscribed in a circle is a square.\vspace{5mm}\\
\centering \large\textsc{2  S}\footnotesize\textsc{OLUTION}\vspace{5mm}\\
\raggedright Let the radius of circle be r = 5cm with center as origin and let the coordinates of the rectangle be A,B,C and D such that,
	\begin{gather*}
		\vec{A} = \myvec{rcos(\theta_1)\\rsin(\theta_1)}\hspace{2mm},\hspace{2mm}\vec{B} = \myvec{-rcos(\theta_2)\\rsin(\theta_2)}\hspace{2mm},\hspace{2mm}\vec{C} = -\vec{A}\hspace{2mm},\hspace{2mm}\vec{D} = -\vec{B}\\
		\theta_1 = 45\degree\\
		\theta_2 = 60\degree
	\end{gather*}
\raggedright The equation of circle is represented as,
	\begin{gather}
		\vec{x^T}\vec{x} + f = 0
	\end{gather}
	\centering{where, f  =-25}\vspace{2mm}
\raggedright As the coordinates of rectangle lies on circumference of  circle, they satisfy the above equation.\\
	\begin{gather*}
		\vec{A^T}\vec{A} + f = \myvec{\frac{5}{\sqrt{2}} & \frac{5}{\sqrt{2}}}\myvec{\frac{5}{\sqrt{2}} \\ \frac{5}{\sqrt{2}}} -25\\
		 = 25 -25 \\
	         = 0
	\end{gather*}
	\begin{gather*}
		\vec{B^T}\vec{B} + f = \myvec{\frac{5}{2} & \frac{5\sqrt{3}}{2}}\myvec{\frac{5}{2} \\ \frac{5\sqrt{3}}{2}} -25\\                                                                                                        = \frac{25}{4} + \frac{75}{4} -25 \\
		= 0
          \end{gather*}
	  \raggedright{Similarly, C and D also satisfies the above equation}\vspace{2mm}
	\raggedright{And,}
	\begin{gather}
		\vec{(A-B)^T}\vec{(C-B)} = \myvec{\frac{5}{\sqrt{2}}-\frac{5}{2} & \frac{5}{\sqrt{2}}-\frac{5\sqrt{3}}{2}}\myvec{\frac{-5}{\sqrt{2}}-\frac{-5}{2} \\ \frac{-5}{\sqrt{2}}-\frac{-5\sqrt{3}}{2}}\\
		\implies{\vec{(A-B)^T}\vec{(C-B)} = 0}
	\end{gather}
	\raggedright{And,}
	\begin{align*}
		\vec{C-B} = \myvec{\frac{-5}{\sqrt{2}} \\ \frac{-5}{\sqrt{2}}} - \myvec{\frac{5}{2} \\ \frac{5\sqrt{3}}{2}}\\
		\implies{\vec{C-B} = \myvec{\frac{-5}{\sqrt{2}}-\frac{5}{2} \\ \frac{-5}{\sqrt{2}}-\frac{5\sqrt{3}}{2}}}\\
		\vec{D-A} = \myvec{\frac{-5}{2} \\ \frac{-5\sqrt{3}}{2}} - \myvec{\frac{5}{\sqrt{2}} \\ \frac{5}{\sqrt{2}}}\\
		\implies{\vec{D-A} = \myvec{\frac{-5}{2}-\frac{5}{\sqrt{2}} \\ \frac{-5\sqrt{3}}{2}-\frac{-5}{\sqrt{2}}}}
	\end{align*}
	\raggedright{Hence,}
	\begin{gather}
		\vec{(C-B)} = \vec{(D-A)}
	\end{gather}
	\raggedright{As the diagonals of the rectangle bisect each other, let the intersection point of the diagonals be point P, such that it is equidistant from the coordinates A,B,C and D.}
	\begin{gather*}
		\vec{P} = \frac{\vec{(A+C)}}{2}\\
		\vec{P} = \myvec{0\\0}
	\end{gather*}
	\raggedright{Hence, point P is center of circle.}
	\begin{gather*}
 \| P-A \| = \| P-C \| = 5
	\end{gather*}
	\raggedright{Hence, the distance from P to A and C is equal to the radius.}\\
	\raggedright{Therefore, the diagonal of the rectangle is equal to the diameter of the circle.}\vspace{10mm}\\
	\raggedright{Let the sides of the rectangle be x and y, by pythagoras theorem,} 
	\begin{gather}
	x^2 + y^2 = (2r)^2\\
		\implies{y = \sqrt{4r^2 - x^2}}
	\end{gather}
	Area of rectangle = xy\vspace{2mm}\\
	Area of rectangle = x$\sqrt{4r^2-x^r}$\vspace{2mm}\\
Solving using Gradient ascent, to get maximum area
\begin{align*}
x_{n+1} &= x_n + \alpha \nabla f(x_n) \\
	\implies x_{n+1} &= x_n + \alpha \nabla f(x\sqrt{4r^2-x^2})
\end{align*}
Taking $x_0 = 1, \alpha = 0.001$ and $precision = 0.00000001$, values obtained using python are:
\begin{align}
\boxed{\text{Maxima} = 49.99} \\
\boxed{\text{Maxima Point} = 7.07}
\end{align}
	As,the maximum point is x, when substituted in eq(2), we get\\
	\centering y = 7.07\\
Hence, the sides of the rectangle are equal and forms a square of maximum area.
\vspace{5mm}\\
\begin{mdframed}
\raggedright\large{https://github.com/madind5668/FWC\\/blob/main/optimization/advanced\\/codes/main.py}
\end{mdframed}
\end{multicols}
\end{document}

