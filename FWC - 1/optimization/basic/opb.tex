\documentclass[a4paper,10pt]{report}
\usepackage[latin1]{inputenc}
\usepackage{amsmath}
\usepackage{amsmath,bm}
\usepackage{amsthm}
\usepackage{mathtools}
\usepackage{amsfonts}
\usepackage{amssymb}
\usepackage{graphicx}
\usepackage{array}
\usepackage{booktabs}
\usepackage{hyperref}
\usepackage{multicol}
\usepackage[margin=0.5in]{geometry}
\usepackage{karnaugh-map}
\usepackage[framemethod=tikz]{mdframed}
\usepackage{makecell}
\newcommand{\myvec}[1]{\ensuremath{\begin{pmatrix}#1\end{pmatrix}}}
\let\vec\mathbf
\newcommand{\mydet}[1]{\ensuremath{\begin{vmatrix}#1\end{vmatrix}}}
\providecommand{\mbf}{\mathbf}
\providecommand{\pr}[1]{\ensuremath{\Pr\left(#1\right)}}
\providecommand{\qfunc}[1]{\ensuremath{Q\left(#1\right)}}
\providecommand{\sbrak}[1]{\ensuremath{{}\left[#1\right]}}
\providecommand{\lsbrak}[1]{\ensuremath{{}\left[#1\right.}}
\providecommand{\rsbrak}[1]{\ensuremath{{}\left.#1\right]}}
\providecommand{\brak}[1]{\ensuremath{\left(#1\right)}}
\providecommand{\lbrak}[1]{\ensuremath{\left(#1\right.}}
\providecommand{\rbrak}[1]{\ensuremath{\left.#1\right)}}
\providecommand{\cbrak}[1]{\ensuremath{\left\{#1\right\}}}
\providecommand{\lcbrak}[1]{\ensuremath{\left\{#1\right.}}
\providecommand{\rcbrak}[1]{\ensuremath{\left.#1\right\}}}
\begin{document}
\raggedright{\includegraphics[scale=0.07]{logo.jpg}}\hspace{12.425cm}\raggedleft FWC22025\vspace{2mm}\\
\centering\Large\textbf{OPTIMIZATION-BASIC}\vspace{5mm}
\begin{multicols}{2}
\centering \large\textsc{C}\footnotesize\textsc{ONTENTS}\vspace{5mm}\\
\raggedright\large\textbf{1\hspace{1cm}Problem}\hspace{5.2cm}1\vspace{5mm}\\
\raggedright\large\textbf{2\hspace{1cm}Solution}\hspace{5.25cm}1\vspace{5mm}\\
\centering \large\textsc{1  P}\footnotesize\textsc{ROBLEM}\vspace{5mm}\\
\raggedright\large{A farmer mixes two brands P and Q of cattle feed. Brand P, costing Rs 250 per bag, contains 3 units of nutritional element A, 2.5 units of element B and 2 units of element C. Brand Q, costing Rs 200 per bag contains 1.5 units of nutritional element A, 11.25 units of element B and 3 units of element C. The minimum requirements of nutrients A,B and C are 18 units, 45 units and 24 units respectively. Determine the number of bags of each brand which should be mixed in order to produce a mixture having minimum cost per bag? What is the minimum cost of the mixture per bag?}\vspace{5mm}\\

\centering \large\textsc{2  S}\footnotesize\textsc{OLUTION}\vspace{5mm}\\
\raggedright\large{Let x and y be the number of bags of brand P and Q respectively. Obviously $x\ge0$, $y\ge0$. Mathematical formulation of the given problem is as follows:}\\
\begin{align}
Minimize Z = \min_{x,y}250x+200y (Cost per bag)
\end{align}
Subject to the constraints:\\
constraint on element A
\begin{gather}
\hspace{1mm}3x+1.5y \ge 18  \implies{2x+y \ge 12}
\end{gather}
constraint on element B
\begin{gather}
\hspace{1mm}2.5x+11.25y \ge 45  \implies{2x+9y \ge 36}
\end{gather}
constraint on element C
\begin{gather}
\hspace{1mm}2x+3y \ge 24
\end{gather}

which can be expressed in vector form as
\begin{gather}
	Z = \min_{\vec{x}}\myvec{250 &200}\vec{x}
\end{gather}
\begin{gather}
	\myvec{2&1\\2&3\\2&9}\vec{x}=\myvec{12\\24\\36}
\end{gather}

Solving using cvxpy,we get
\begin{gather}
	Z_{min} = 1950\\
	\vec{x}=\myvec{3\\6}
\end{gather}

Hence, the minimum cost is Z=1950 which occurs at $\myvec{3\\6}$\\
Thus, the farmer should produce a mixture of cattle feed with 3 bags of brand P and 6 bags of brand Q to have minimum cost per bag.\\

\begin{mdframed}
\raggedright\large{https://github.com/madind5668/FWC\\/blob/main/optimization/basic\\/codes/main.py}
\end{mdframed}
\end{multicols}
\end{document}
